\chapter{Conclusions and Future Work}
\label{chp:conclusionsandfuturework}

\section{Conclusions}
% introduction
This thesis proposed a new method for activity detection while only using visible light and a photodiode with the goal of saving energy. This can be achieved by measuring reflections of light with a photodiode, produced by a light. A model has shown that the changes in reflection are measurable by a photodiode if a person walks by.

A flash does not appear as a perfect square to a photodiode. Therefore, a method was created to extract the most relevant features from a flash. These features where then analysed by a newly created algorithm capable of analysing consecutive flashes.

A prototype has been made with off-the-shelve parts, implementing the proposed system. The prototype was created to work in a hallway of an university. It showed to detect 100\% of all persons directly walking underneath the light, no matter the colour of their clothing or environment the system was placed in. 86.5\% and 76.5\% of pedestrians where detected at 30 and 60 cm from the centre of the set-up. The system has a 25\% chance on a false detection every minute, which is caused by poor build quality of the prototype.

Even though the prototype has a relatively high false positive ratio, it serves as a good proof of concept for detecting activity with short flashes of visible light.

\section{Future Work}
The presented work serves as a proof of concept for detecting activity in the line of sight of an LED. I personally think that the potential of this system is huge, especially if a dedicated platform is created, as most shortcomings of the current prototype can be solved if a better platform is built. Below I have listed several ideas for future research.

\begin{itemize}
	\item \textbf{Multiple units in one room} - The algorithm is currently designed for a stand-alone device. If we would hang multiple of these systems in the same room then it's likely that some of the light flashes overlap and trigger a false positives regularly. This problem could be solved by having each detector flash in another timeslot, but this requires more research.
	\item \textbf{Tracking} - The system is currently only detecting activity. It could also be expanded for other purposes. It might for example be possible to use multiple photo diodes, lenses or field of view blockers to track bypassing pedestrians.
	\item \textbf{A dark sensing network} - Multiple working units in one room is nice, but multiple units working together to track, predict and illuminate the path of a pedestrian is nicer. This could be achieved by having the devices communicate using the flashes already generated by the device (visible light communication).
\end{itemize}