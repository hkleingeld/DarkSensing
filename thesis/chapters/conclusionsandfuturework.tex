\chapter{Conclusions and Future Work}
\label{chp:conclusionsandfuturework}

\section{Conclusions}
TODO CONCLUSIONS
 - It works with crappy hardware
 - 

	

\section{Future Work}
The present work severs as a proof of concept for detecting activity in the line of sight of an LED. I personally think that the potential of this system is huge, especially if a dedicated platform is created. Below I have listed several ideas for future research, which I think have great potential, if a proper platform is created and it would be awesome if anyone would the dark sensing project to the next level.

\begin{itemize}
	\item \textbf{Multiple units in one room} - The algorithm is currently designed for a stand-alone device. If we would hang multiple of these systems in the same room then it's likely that some of the light flashes overlap and trigger a false positives regularly. This problem could be solved by having each detector flash in another timeslot, but this requires more research.
	\item \textbf{Tracking} - The system is currently only detecting activity. It could also be expanded for other purposes. It might for example be possible to use multiple photo diodes, lenses or field of view blockers to track bypassing pedestrians.
	\item \textbf{A dark sensing network} - Multiple working units in one room is nice, but multiple units working together to track, predict and illuminate the path of a pedestrian is nicer. This could be achieved by having the devices communicate using the flashes already generated by the device (visible light communication).
\end{itemize}