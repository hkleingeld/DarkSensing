\chapter{Introduction}
\label{chp:introduction}
Nowadays, 19\% of the global energy consumption is used for lighting. For this reason, saving energy in lighting is vital. A simple way to save energy is to “simply” turn the lights off, or reduce the amount of light used when nobody is around. This thesis proposes a new method for luminaires to detect the presence of humans or objects using only the light it emits and a photodiode. This method can then be used to control the light output of a luminaire based on if somebody is detected or not and save energy by turning the light off.

test test

\section{Problem statement}
\label{Problem statement}
Is it possible to create a system, that can detect the activity of humans or objects by measuring reflections of visible light while being invisible to the human eye?
\begin{itemize}\itemsep2pt
	\item How strong is a reflection obtained from a flash in a realistic scenario and how much does this reflection vary if a human is in the area?
	\item What are the challenges in obtaining reflections when the light is turned on for a very short time and how can they be tackled?
	\item What additional signals are received by the system (beside the reflection of the flash) and what algorithm can be used to convert the received signal in a reliable logical signal: Detection or no detection?
\end{itemize}
\section{Contributions}
\label{sec:Contributions}

\section{Organization}
\label{sec:Organization}
The organization of the thesis as follows: Chapter 2 provides the necessary background information required to understand the thesis. Chapter 3 shows related work. The 4th chapter explains how dark sensing was developed and chapter 5 shows the building of a realistic dark sense device. The final chapter discusses the performance of the prototype and points out future work.