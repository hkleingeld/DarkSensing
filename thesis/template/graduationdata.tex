%De aankondiging bevat de spreker, titel, plaats, datum en tijd, samenstelling van de afstudeercommissie en een korte samenvatting (maximaal 25 regels).
\thispagestyle{empty}

\noindent \textbf{Author}\\
\begin{tabular}{l}
\reportAuthor{} (\reportUrlEmail)\\
\end{tabular}\\
\noindent \textbf{Title}\\
\begin{tabular}{l}
\reportTitle\\
\end{tabular}\\
\noindent \textbf{MSc presentation}\\
\begin{tabular}{l}
% <MM> DD, YYYY (like \today)
\presentationDate\\
\end{tabular}

\vspace{1.1cm}

\noindent \textbf{Graduation Committee}\\
\begin{tabular}{ll}
\graduationCommittee
\end{tabular}


\begin{abstract}
%What is the problem + why is it important
Nowadays, 19\% of the global energy consumption is used for lighting. For this reason, saving energy in lighting is vital. A simple way to save energy is to “simply” turn the lights off, or reduce the amount of light used when nobody is around. This thesis proposes a new method for luminaires to detect the presence of humans and objects which only uses a photodiode and a fraction of the light a luminaire normally emits, namely \textit{Dark Sensing}.

%What is my solution
Dark sensing works by sending out short flashes of light. These short flashes use little energy and are barely visible to the user. These flashes get reflected by the environment and received by a photodiode placed next to the light. By extracting a key feature of the received flash, we obtain a metric representing the surrounding area. If an object enters the observed area, the reflections of light will change. These changes will be noticed by the system, which triggers a detection resulting in the light being turned on.

%What follows from my solution
A prototype was created which shows the potential of the newly developed method. The prototype was tested in two different environments and detects between 73\% and 90\% of bypassing pedestrians, depending on the accepted false positive ratio (0 to 0.05).

\setcounter{page}{3}
\reportAbstract{}
\end{abstract}

\clearpage
